%\documentclass[mathserif]{beamer} %
%\usepackage{eulervm}
%\usetheme{metropolis}

\documentclass{beamer} %
\usetheme{Klope}
%\usepackage[absolute,overlay]{textpos}
\usepackage{textpos}
\usepackage{mdframed}
\newcommand\FrameText[1]{%
  \begin{textblock*}{\paperwidth}(12pt,\textheight)
    \raggedright #1\hspace{.1em}
  \end{textblock*}}
\usepackage{color}


\usepackage{etex}
\usepackage{graphicx}
%\usepackage[language=eng]{babel}
\usepackage{cite}
%\usepackage{qcircuit}
\usepackage{bbold}
\usepackage{lmodern}
\usepackage[utf8]{inputenc}
%\usepackage[font=small,labelfont=bf]{caption}
%\usepackage[font=small,labelfont=bf]{subcaption}
%\DeclareCaptionFont{tiny}{\tiny}
%\usepackage{caption, subcaption}
\usepackage{amssymb}
\usepackage{amsmath}
\DeclareMathOperator{\Tr}{Tr}
\usepackage{braket}
%\usepackage{physics}
\newcommand\abs[1]{\left|#1\right|}
\graphicspath{ {./images/} }
\usepackage{tikz}
\usepackage{placeins}
%\linespread{1.005}\selectfont
\usepackage{paralist}
\usepackage{tabto}
\setlength{\parindent}{0mm}
\setbeamercovered{transparent}
%\usefonttheme[onlymath]{serif}

\title{Algoritmi genetici in Mathematica: soluzione del problema di Koza}
%\subtitle{Using Beamer}
\author{Claudio Chiappetta}
\institute{Università degli studi di Milano}
\date{}%\today

\begin{document}
\setbeamercolor{background canvas}{bg=white}
\setbeamercolor{block body}{bg=white}
\setbeamercolor{block title}{bg=white}



\begin{frame}
\titlepage 
\end{frame}

\section{Il problema}
\begin{frame}
\frametitle{Il problema}

\begin{itemize}
\item Parola bersaglio: \emph{"universal"}
\item Due insiemi di lettere: una pila e un insieme non ordinato
\item Abbiamo un set di azioni
\item Trovare una combinazione di azioni che formi la parola bersaglio sulla pila
\end{itemize}
\end{frame}

\section{L'approccio}
\begin{frame}
\frametitle{L'approccio}

Due metodi: 
\begin{itemize}
\item Un individuo è un comando di Mathematica
\item Un individuo è un albero (lista annidata di funzioni)
\end{itemize}

\begin{block}{Problema: Controllare l'esecuzione degli individui}
Si è scelto di usare dei nomi "finti" per le funzioni
\end{block}

\end{frame}

\begin{frame}
\frametitle{Generazione di individui casuali}

\begin{itemize}
\item Si parte da \texttt{EQ[\#1,\#2] \&}
\item Una funzione ricorsiva inserisce come argomenti due comandi casuali, e chiama se stessa per riempire gli eventuali argomenti
\end{itemize}

\end{frame}

\begin{frame}
\frametitle{Esecuzione di un individuo}
La struttura ad albero viene trasformata ricorsivamente in un espressione di Mathematica
\begin{itemize}
\item Una funzione trasforma un nodo in un comando di Mathematica, e ne determina gli argomenti applicando se stessa ai sottorami.
\item Top-down, backtracking
\end{itemize}
\end{frame}



\begin{frame}
Nel comando con nomi inattivi, questi vengono sostituiti con le vere funzioni utilizzando \texttt{ReplaceAll}. 


\frametitle{Esecuzione di un individuo}
A questo punto l'espressione è immediatamente valutabile


\begin{block}{Problema: la funzione \texttt{DU}}
È necessario valutare più di una volta gli argomenti. Si risolve utilizzando opportunamente \texttt{HoldAll} 
\end{block}

\end{frame}

\begin{frame}
\frametitle{Fitness}
\begin{itemize}
\item Funzioni che non premiano l'efficienza
\begin{itemize}
\item Funzione standard
\item Funzione generosa
\end{itemize}
\end{itemize}

\begin{itemize}
\item Funzioni che premiano l'efficienza
\begin{itemize}
\item Funzione di Koza
\item Variazione della funzione di Koza
\end{itemize}
\end{itemize}


\end{frame}

\begin{frame}
\frametitle{Crossover}
\begin{itemize}
\item Presa una coppia di genitori, si stacca un ramo da ognuno dei due
\item Si attacca su uno il ramo dell'altro
\end{itemize}

\end{frame}



\begin{frame}
\frametitle{Crossover: Individui del secondo tipo}
\begin{itemize}
\item Estrarre un punto casuale in una lista di cui non si conosce lunghezza nè profondita
\end{itemize}

\end{frame}

\begin{frame}
\frametitle{Crossover: Individui del secondo tipo}
\begin{block}{Reservoir sampling}
\begin{itemize}
\item L'algoritmo prende in ingresso un nodo dell'albero
\item Lo mette nella riserva con probabilità $1/passi$, dove $passi$ è il numero corrente di passi
\item Chiama se stesso sui sottorami del nodo corrente
\item Dopo che ha processato l'ultima foglia ritorna la riserva
\item Top-down, backtracking
\item $O(n)$, dove $n$ è il numero di nodi 
\end{itemize}
\end{block}
\end{frame}

\begin{frame}
\frametitle{Crossover: Individui del secondo tipo}
\begin{itemize}
\item L'inserzione di un ramo un un punto casuale è più complicata dell'estrazione
\item Cambiare la gestione della riserva
\end{itemize}

\end{frame}

\begin{frame}
\frametitle{Crossover: Individui del primo tipo}
\begin{block}{Estrazione}

\begin{itemize}
\item \texttt{RandomChoice}, \texttt{Level}
\item Una riga di codice
\end{itemize}

\end{block}

\begin{block}{Inserzione}

\begin{itemize}
\item \texttt{RandomChoice}, \texttt{Level}, \texttt{Position} e \texttt{ReplacePart} 
\item $3$ righe  di codice
\end{itemize}

\end{block}
\end{frame}


\begin{frame}
\frametitle{Riproduzione}
\begin{itemize}
\item Crossover su $N_{individui}-2$ elementi
\item Vengono mantenuti due individui con fitness massima
\item Mutazione
\end{itemize}
\end{frame}


\section{Ottimizzazione e risultati}
\begin{frame}
\frametitle{Ottimizzazione}
\begin{itemize}
\item $p_c \sim 0.7$
\item $p_m \lesssim 0.01$ (alta)
\end{itemize}

\begin{block}{Problema}
Vengono generati individui con loop infiniti. Si usano i metodi utilizzati da Koza per ottenere programmi efficienti
\end{block}
\end{frame}

\begin{frame}
\frametitle{Risultati}
\begin{itemize}
\item Programmi funzionanti con lunghezza arbitraria vengono trovati prima della generazione $30$.
\item Si ottengono programmi che terminano dopo un numero di passi finito solo utilizzando una variazione della fitness di Koza
\item In questo caso, si trovano programmi funzionanti fra la generazione $30$ e la $50$; programmi ragionevolmente efficienti compaiono dopo la generazione $60$
\item Tempo di esecuzione altamente variabile, dell'ordine di $1 \div 10$ ore.
\end{itemize}

\end{frame}


\section{Grazie per l'attenzione}
\end{document}