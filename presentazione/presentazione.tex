%\documentclass[mathserif]{beamer} %
%\usepackage{eulervm}
%\usetheme{metropolis}

\documentclass{beamer} %
\usetheme{metropolis}
\usepackage[absolute,overlay]{textpos}
\newcommand\FrameText[1]{%
  \begin{textblock*}{\paperwidth}(12pt,\textheight)
    \raggedright #1\hspace{.1em}
  \end{textblock*}}
\usepackage{color}


\usepackage{etex}
\usepackage{graphicx}
%\usepackage[language=eng]{babel}
\usepackage{cite}
%\usepackage{qcircuit}
\usepackage{bbold}
\usepackage{lmodern}
\usepackage[utf8]{inputenc}
%\usepackage[font=small,labelfont=bf]{caption}
%\usepackage[font=small,labelfont=bf]{subcaption}
%\DeclareCaptionFont{tiny}{\tiny}
%\usepackage{caption, subcaption}
\usepackage{amssymb}
\usepackage{amsmath}
\DeclareMathOperator{\Tr}{Tr}
\usepackage{braket}
%\usepackage{physics}
\newcommand\abs[1]{\left|#1\right|}
\graphicspath{ {./images/} }
\usepackage{tikz}
\usepackage{placeins}
%\linespread{1.005}\selectfont
\usepackage{paralist}
\usepackage{tabto}
\setlength{\parindent}{0mm}
\setbeamercovered{transparent}
%\usefonttheme[onlymath]{serif}

\title{Algoritmi genetici in Mathematica: soluzione del problema di Koza}
%\subtitle{Using Beamer}
\author{Claudio Chiappetta}
\institute{Università degli studi di Milano}
\date{}%\today

\begin{document}
\setbeamercolor{background canvas}{bg=white}
\setbeamercolor{block body}{bg=white}
\setbeamercolor{block title}{bg=white}



\begin{frame}
\titlepage 
\end{frame}

\section{Il problema}
\begin{frame}
\frametitle{Il problema}

\begin{itemize}
\item Parola bersaglio: \emph{"universal"}
\item Due insiemi di lettere: una pila e un insieme non ordinato
\item Abbiamo un set di azioni
\item Trovare una combinazione di azioni che formi la parola bersaglio sulla pila
\end{itemize}
\end{frame}

\section{L'approccio}
\begin{frame}
\frametitle{Usando gli strumenti di Mathematica}
\begin{itemize}
\item Un individuo è un comando di Mathematica
\item 
\end{itemize}
\end{frame}


\end{document}